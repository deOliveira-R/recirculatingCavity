% !TEX root = ../article.tex

\subsection{Post processing specification}

Field data for velocity, pressure, temperature and flux are shown in figures \ref{fig:flux0}-\ref{fig:T} and used as a qualitative demonstration of the shape of the fields.
The fields for the analytical solutions will not be shown because there is no visual difference from the numerical solution fields.

The absolute error field is calculated subtracting the numerical solution field from the analytical solution field taking the absolute value as $ \abs{ r_{n} - r_{ n,\text{ref} } } $, where $ r_{n} $ is the numerical value of a response in cell n, and $ r_{ n,\text{ref} } $ is the analytical value of the response in the same cell.
The error fields will not be shown for brevity.

Plots over line AA' shown in figure \ref{fig:domain} for all numerical, analytical and their respective relative error ($ \abs{ r_{n} - r_{ n,\text{ref} } }/r_{ n,\text{ref} } \cdot 100\% $) will be shown, allowing a quantitative evaluation of the results.

In addition, $ \text{L}_{1} $ and $ \text{L}_{2} $ norms are calculated as shown in equations \ref{eqn:L1norm} and \ref{eqn:L2norm}, where $ \omega_{n} $ is the cell volume and $ \Omega $ is the domain volume.

\begin{equation}
    \LOneNorm
\end{equation}

\begin{equation}
    \LTwoNorm
\end{equation}

As can be seen, the $ \text{L}_{1} $ norm represents the mean volume-weighted absolute error, and the $ \text{L}_{2} $ norm the root-mean-square of the volume-weighted absolute error.
For the particular case of uniform meshes, as used in this study, the equations for the norms degenerate into simple averages over the cells.
Nonetheless, the equations are coded as presented, accounting for possible future developments using a non-uniform grid.
These norms are used for a sensitivity study of the mesh and to evaluate mesh convergence.
